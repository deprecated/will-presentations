\documentclass[presentation]{beamer}
\usepackage{will-beamer-puebla} 
\tolerance=1000

\graphicspath{ 
  {figs/},
  {../TownsvillePoster/figs/},
  {../LarimTalk2010/movies/},   
}

\setkeys{Gin}{width=\linewidth, height=0.8\textheight, keepaspectratio=true}

% \setbeameroption{show notes}

% \renewcommand\baselinestretch{1.2}

\title[Ionized gas dynamics]{Dynamics of ionized gas around\\ massive young star clusters}
\author{\textit{William J. Henney}}
\date[Baltimore 2013]{
  December 2013 \(\cdot\) Puebla, Mexico
  \par\bigskip
  \alert{\textit{Remember to turn off power saver!}}
}

\institute[CRyA, UNAM]
{
  \structure{Centro de Radioastronomía y Astrofísica\\
    UNAM, Morelia, México}
}

\hypersetup{
  pdfkeywords={Massive Star Clusters, Astrophysics, Dynamics, Radiation},
  pdfsubject={},
  pdfcreator={Lovingly hand-crafted by the author using pdflatex and beamer}
}


\AtBeginSection[]
{
  \begin{frame}<beamer>
    \frametitle{Coming up next \dots}
    \tableofcontents[
    sectionstyle=show/shaded,
    currentsubsection, 
    hideothersubsections
    ]
  \end{frame}
}



\newcommand\missingmovietext{
  \begin{center}
    \bfseries\ttfamily
    This PDF viewer does not support embedded videos
    \par \bigskip
    To view the movie, please open the PDF file in \textit{Adobe Reader}
    
  \end{center}
  }

\begin{document}

\maketitle

\begin{frame}
\frametitle{Principal collaborators}
\par\medskip
\begin{block}{CRyA-UNAM, Morelia, Mexico \dots}
\begin{description}
\item[\small HD] \textit{Jane Arthur}
\item[\small Turbulence] \textit{Enrique Vázquez-Semadeni}
\item \textit{Sac-Nicté Serrano Medina} (MSc)
\item[\small Bowshocks] \textit{A. Jorge Tarango Yong} (MSc/PhD)
\item \textit{Luis Ángel Gutierrez Soto} (MSc)
\end{description}
\end{block}

\begin{block}{Elsewhere in the world \dots}
  \begin{description}
  \item[\small MHD] \textit{Fabio de Colle} (ICN-UNAM, Mexico)
  \item[\small Radiation] \textit{Garrelt Mellema} (Stockholm Observatory, Sweden)
  \item[\small \textmu{}-physics] \textit{Gary Ferland} (Kentucky, USA)
  \item[\small Observations] \textit{María Teresa García-Díaz} (IA-UNAM, Ensenada, Mexico)
  \item \textit{Bob O'Dell} (Vanderbilt, USA)
    \end{description}
\end{block}

\end{frame}

\begin{frame}<beamer>
  \frametitle{Dynamics of the Orion Nebula}
  \tableofcontents[hidesubsections]
\end{frame}

\section{Whatever}
\subsection{Turbulent models}

\newlength\maxheight
\setlength\maxheight{0.8\textheight}
\newlength\moviewidth
\setlength\moviewidth{0.7\textwidth}
\newlength\movieheight

\begin{frame}
\frametitle{Turbulent models: initial conditions}
\includegraphics{poland-figs/rgb-CPF-initial}
\end{frame}

\begin{frame}[shrink=5]
\frametitle{Turbulent models: state of play}
\begin{block}{Physics we have}
  \begin{itemize}
  \item 3D time-dependent, hydrodynamics
  \item Approximate radiative transfer
  \item Microphysics:
    \begin{itemize}
    \item good for ionized gas
    \item fair for PDR
    \item poor for molecular gas
    \end{itemize}
  \item {}[Ideal magnetohydrodynamics]
  \end{itemize}
\end{block}
\begin{block}{Physics we lack}
  \begin{itemize}
  \item Stellar winds
  \item Radiation pressure
  \item Diffuse field
  \item Self-gravity
  \item {\footnotesize Better microphysics, better radiative transfer,
    \scriptsize multifluids, non-ideal MHD, \tiny \(\upkappa\)-distributions, etc \dots}
  \end{itemize}
\end{block}
\end{frame}

\begin{frame}[plain]%%
  \newlength\figwidth
  \setlength\figwidth{0.33\textwidth}
  \renewcommand\arraystretch{0.0}
  \setlength\tabcolsep{0pt}
  \graphicspath{
    {poland-figs/movie-stills/O-Star-512-PDR-2012/},
    }
  \begin{tabular}{lll}
    \includegraphics[width=\figwidth]{01-Opening-Titles}
    & 
    \includegraphics[width=\figwidth]{02-Model-Parameters}
    & 
    \includegraphics[width=\figwidth]{03-Color-Scheme}
    % & 
    % \includegraphics[width=\figwidth]{04-Evolution-Start}
    \\
    \includegraphics[width=\figwidth]{05-Evolution-Mid}
    & 
    \includegraphics[width=\figwidth]{06-Evolution-End}
    & 
    \includegraphics[width=\figwidth]{07-Detail-View}
    % & 
    % \includegraphics[width=\figwidth]{08-Swimming-Sisters} 
    \\
    % \includegraphics[width=\figwidth]{09-Long-Wav-Color-Scheme}
    % & 
    \includegraphics[width=\figwidth]{10-Long-Wav-Mid}
    & 
    \includegraphics[width=\figwidth]{11-Long-Wav-End}
    & 
    \includegraphics[width=\figwidth]{12-Simulations-Credit}
\end{tabular}

\end{frame}


\begin{frame}
  \frametitle{Turbulent models: results}
  \begin{columns}
    \column{0.6\linewidth}
    \begin{itemize}
    \item Many morphological features of observed \hii{} regions are
      reproduced naturally
      \begin{itemize}
      \item Due to existing density structure in the
        turbulent molecular cloud, combined with fragmentation induced
        by interaction with the ionized gas
      \end{itemize}
    \item Velocity dispersions of order the sound speed are
      maintained in the ionized gas during the entire evolution
    \item The highest pressure neutral/molecular gas is driven to
      equipartition between thermal, magnetic, and turbulent energies
    \item Lower pressure gas bifurcates into zones dominated by one or
      the other
    \end{itemize}
    \column{0.4\linewidth}
    \includegraphics{poland-figs/comparison3_vs_t_Ostar}%
  \end{columns}
\end{frame}

\begin{frame}
  \frametitle{Turbulent models: more results}
  \begin{columns}
    \column{0.5\linewidth}
    \includegraphics{poland-figs/mhd-pressures-rgb-Ostar-et-0200-pram-pmag}
    \column{0.5\linewidth}
    \includegraphics{poland-figs/mhd-pressures-rgb-Ostar-et-0200-n-B}
  \end{columns}
\end{frame}

% This doesn't work with multimedia package since movie is opaque even
% in Preview.app
% \usebackgroundtemplate{
%   \parbox[c][\paperheight][c]{\paperwidth}{\missingmovietext}
% }

\def\MovieFile{poland-figs/O-Star-512-PDR-2012.mov}
% Ratio: 0.75
% \setlength\moviewidth{1.27968\paperheight}
% \setlength\movieheight{0.96\paperheight}
\begin{frame}%%
  \frametitle{Turbulent \hii{} regions: the movie}
  \graphicspath{
    {poland-figs/movie-stills/O-Star-512-PDR-2012/},
  }
  \begin{columns}
    %%
    %% The movie itself
    %%
    \column{0.7\linewidth}
    \setlength\moviewidth{\linewidth}
    \setlength\movieheight{0.75\moviewidth}
    \movie[width=\moviewidth, height=\movieheight, label=bigmovie,
    autostart, showcontrols, start=2s]
    {\includegraphics[width=\moviewidth, height=\movieheight]{01-Opening-Titles}}
    {\MovieFile}
    %%
    %% Buttons to control the movie
    %%
    \column{0.3\linewidth}
    \includegraphics[width=\linewidth]{02-Model-Parameters}
    \par\bigskip
    % Jump to Evolution
    \hyperlinkmovie[start=20s, duration=6s, loop]{bigmovie}
    {\beamerbutton{Evolution of optical line emission}}
    % Jump to 100,000
    \hyperlinkmovie[start=48s, duration=8s, palindrome]{bigmovie}
    {\beamerbutton{Rotation at \(t=100,000\)~years}}
    % Jump to 200,000
    \hyperlinkmovie[start=75s, duration=8s, palindrome]{bigmovie}
    {\beamerbutton{Rotation at \(t=200,000\)~years}}
    % Jump to Long wavelength
    \hyperlinkmovie[start=110s, duration=5s, loop]{bigmovie}
    {\beamerbutton{Neutral/molecular gas evolution}}
    % Jump to Credits 
    \hyperlinkmovie[start=153s]{bigmovie}
    {\beamerbutton{Show the credits}}
    % Open in viewer
    \href{run:\MovieFile}{\beamerbutton{Open in external viewer}}
  \end{columns}
\end{frame}



\end{document}
